% Chapter 4

\chapter{Évaluation et Feedback humain} % Chapter title

\label{ch:3} % For referencing the chapter elsewhere 

\section{Critères et méthodes d'évaluations}


\subsection{Le test de magistrature 2022}

\begin{longtable}{|p{0.7\textwidth}|p{0.3\textwidth}|}
\hline
\textbf{Question} & \textbf{Catégorie} \\
\hline
Peut-on constituer un prévenu gardien d'un objet saisi ? Dans la négative, donnez nous trois raisons. & procédure pénale \\
\hline
Les expressions, auteur présumé de l'infraction, l'inculpé, prévenu et condamné, traduisent quelles étapes des instances judiciaires. & procédure pénale \\
\hline
Quelle est la différence entre l'amnistie et la grâce du point de l'organe de décision ? & procédure pénale \\
\hline
Le ministère public peut-il requérir devant le juge le classement sans suite d'un dossier fixé devant le tribunal ? Dans la négative, donnez deux raisons. & procédure pénale \\
\hline
Le ministère public peut-il aussi introduire la procédure de suspicion légitime du tribunal dont il est membre de composition, si oui à quel titre. & procédure pénale \\
\hline
La nationalité congolaise peut-elle être détenue concurremment avec une autre nationalité par un sujet congolais vivant en \ac{rdc} ? Justifiez votre réponse. & droit civil des personnes \\
\hline
La dissolution du mariage par les autorités coutumières ou familiales peut-elle produire d'effets ? Justifiez votre réponse. & droit civil des personnes \\
\hline
Quelles sont les trois formes de testament consacré par la législation congolaise ? Explicitez-les. & droit civil des personnes \\
\hline
Monsieur FULANI marié à Madame SONGOLO décède en laissant derrière lui deux enfants nés avant le mariage, quatre enfants pendant le mariage, trois enfants hors mariage, un enfant adoptif, deux frères et trois soeurs. Le de cujus n'a laissé qu'un seul bien de valeur en l'occurrence un immeuble acheté auprès de l'ex ONL. L'aîné des enfants estime que le bien doit lui revenir. Les enfants nés pendant le mariage soutiennent qu'ils sont des enfants légitimes et peuvent seuls prétendre à l'héritage du de cujus. De leur côté, les frères et soeurs du défunt pensent que l'immeuble laissé par leur cadet ne doit revenir en priorité qu'à la famille. L'épouse du de cujus formule aussi les mêmes prétentions. S'agissant d'un seul bien de valeur, quelles solutions préconisez-vous à ces différentes prétentions ? & droit civil des personnes \\
\hline
Comment qualifie-t-on dans leur ordre progressif de degré de criminalité, les deux extrêmes de la tentative punissable ? & droit pénal général \\
\hline
Un condamné incarcéré à 12 heures du matin, pour subir un jour d'emprisonnement, constate dans sa fiche de libération qu'on l'a fait sortir le lendemain du jour d'incarcération à 14 heures, est-il en droit de se plaindre pour détention illégale ? & droit pénal général \\
\hline
En République Démocratique du Congo, la peine de fouet qui ne pouvait être infligée que par les juridictions indigènes a été supprimée par a- L'accession du Congo à l'indépendance le 30 juin 1960 b- L'accord global et inclusif du Sun City, après l'A.F.D.L. c- Le décret du 18 décembre 1951 avant l'indépendance & droit pénal général \\
\hline
Le calcul de jour de détention d'une personne incarcérée n commence-t-il : 
a- Le jour de la condamnation par le tribunal ? 
b-Le jour où la condamnation est coulée en force de chose jugée ? 
c- Le jour où sa détention est confirmée par la chambre de conseil ? 
d-Le jour où elle est placée sous mandat d'arrêt provisoire ? 
e-Le jour où elle a été privée de sa liberté ? & droit pénal général \\
\hline
En quoi l'amende judiciaire est différente de l'amende transactionnelle ? donnez au moins 4 points de différence. & droit pénal général \\
\hline

\caption{Questions du test de magistrature Congolais 2022}
\label{table:magisture-test-2022}
\end{longtable}


\section{Évaluation des modèles existants}
\section{Évaluation du modèle RAG}
\section{Évaluation du modèle Fine-tuning}
\section{Résultats et perspectives}
