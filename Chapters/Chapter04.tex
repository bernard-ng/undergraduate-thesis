% Chapter 4

\chapter{Conclusion} % Chapter title

\label{ch:4} % For referencing the chapter elsewhere, use \autoref{ch:5} 

En rétrospective, ce travail a permis d'explorer en profondeur l'application des modèles d'intelligence artificielle dans le domaine du droit, avec un accent particulier sur l'utilisation de systèmes \acf{rag}. À travers une méthodologie rigoureuse incluant la collecte, le pré-traitement et la modélisation des données, nous avons démontré l'efficacité et les limites des modèles existants tout en proposant des améliorations significatives via notre modèle développé.

% Les résultats obtenus montrent que notre modèle surpasse les modèles existants dans plusieurs critères d'évaluation, confirmant ainsi l'importance de l'intégration des techniques de récupération de contexte pour améliorer la pertinence et la précision des réponses générées. Cependant, il reste des défis à surmonter, notamment en ce qui concerne la gestion des biais et la compréhension contextuelle fine des textes juridiques.
 
En réfléchissant à l'avenir, plusieurs pistes de développement et de recherche apparaissent prometteuses. L'amélioration continue des modèles de langage, notamment par l'intégration de données spécifiques au domaine juridique, pourrait considérablement enrichir le contexte et la précision des réponses générées. En outre, la combinaison des approches de machine learning traditionnelles avec les techniques de deep learning pourrait donner naissance à des systèmes hybrides encore plus performants et capables de mieux comprendre les nuances juridiques.

Il est également crucial de développer des mécanismes pour identifier et atténuer les biais présents dans les données et les algorithmes, assurant ainsi des systèmes d'\ac{ia} plus équitables et transparents. L'extension des cas d'utilisation des modèles \ac{rag} dans d'autres domaines du droit, comme le droit international, pourrait offrir de nouvelles perspectives et applications innovantes.

L'amélioration de l'interface utilisateur des applications basées sur l'\ac{ia}, pour les rendre plus intuitives et accessibles aux non-spécialistes, facilitera une adoption plus large dans les pratiques juridiques quotidiennes. Enfin, encourager les collaborations entre experts en droit, ingénieurs en intelligence artificielle et autres parties prenantes est essentiel pour développer des solutions véritablement innovantes et adaptées aux besoins réels du secteur juridique.

En conclusion, ce travail ouvre la voie à de nombreuses possibilités pour l'application de l'intelligence artificielle dans le domaine du droit. Il est essentiel de poursuivre les recherches et les développements dans cette direction pour exploiter pleinement le potentiel de ces technologies émergentes et transformer la pratique juridique moderne.
