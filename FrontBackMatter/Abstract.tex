% Abstract

\renewcommand{\abstractname}{Résumé} % Uncomment to change the name of the abstract

\pdfbookmark[1]{Résumé}{Résumé} % Bookmark name visible in a PDF viewer
\addcontentsline{toc}{chapter}{Résumé}
\begingroup
\let\clearpage\relax
\let\cleardoublepage\relax
\let\cleardoublepage\relax

\chapter*{Résumé}
Dans un milieu où l'accès à l'information juridique est fréquemment entravé et où la complexité du vocabulaire légal constitue un obstacle majeur à la compréhension du système juridique de la République Démocratique du Congo, ce projet ambitionne la création et le déploiement d'un chatbot innovant, basé sur un \ac{llm}. L'objectif de cette initiative est de faciliter l'accès aux connaissances juridiques. Employant des techniques d'intelligence artificielle, ce chatbot entend réinterpréter et simplifier le langage juridique, le rendant intelligible pour tous.

Ce travail se concentre sur l'ajustement des réponses générées des \acfp{llm} en utilisant une architecture \ac{rag} et une application web (Juro) servant d'interface aux utilisateurs.

L'évaluation de notre approche sera réalisée via des retours d'experts humains, notamment des juristes, afin de mesurer l'efficacité du chatbot dans la simplification des concepts juridiques.

\vspace{1cm}
\textbf{Mots clés~:} Chatbot, Large Language Model (LLM), Système Juridique Congolais, Vulgarisation, Accessibilité à l'Information, Intelligence Artificielle, Fine-tuning, Retrieval-augmented Generation

\endgroup			

\vfill

\pagebreak



\pdfbookmark[2]{Abstract}{Abstract} % Bookmark name visible in a PDF viewer
\addcontentsline{toc}{chapter}{Abstract}
\begingroup
\let\clearpage\relax
\let\cleardoublepage\relax
\let\cleardoublepage\relax

\chapter*{Abstract}
In an environment where access to legal information is frequently hampered, and where the complexity of legal vocabulary is a major obstacle to understanding the legal system in the Democratic Republic of Congo, this project aims to create and deploy an innovative chatbot, based on a \ac{llm}. The aim of this initiative is to facilitate access to legal knowledge, by making legal terms more accessible to the Congolese population. Using artificial intelligence techniques, this chatbot aims to reinterpret and simplify legal language, making it intelligible to all.

This work focuses on the tuning of LLMs to facilitate understanding of the Congolese legal framework, exploiting advanced technologies such as fine-tuning and retrieval-augmented generation \cite{lewis2021retrievalaugmented}. 

The evaluation of our approach will be carried out via feedback from human experts, notably lawyers, in order to measure the chatbot's effectiveness in simplifying legal concepts. This evaluation will involve verifying its ability to break down linguistic and cognitive barriers, thus making legal knowledge accessible to all Congolese citizens, in line with the country's principles of justice and equality.

\vspace{1cm}
\textbf{Key words~:} Chatbot, Large Language Model (LLM), Congolese Legal System, Popularization, Information Accessibility, Artificial Intelligence, Fine-tunig, Retrieval-augmented generation


\endgroup			

\vfill
